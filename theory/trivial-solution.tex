\subsection{Trivial Solution}
The most obvious solution is to compare each coordinate in each set $A\in \mathcal{F}$ with each coordinate in $Q$ to calculate the size of the intersection $|A\cap U|$ and union $|A \cup U|$. This solution will always give the correct answer, and query in $O(d^2n)$ time trivially, but suffers from the \textit{curse of dimensionality}. As the amount of dimensions in the data set doubles, the runtime quadruples! When working with high-dimensional datasets like often seen in text processing, this can have huge consequences. \citet{li2011hashing} has shown that it is easy to reach a dimensionality upwards of $d=2^{83}$ when using \textit{5-shingles} (5 contigous words) of the 10,000 most common English words.
A simple improvement is to make a hash table for $D$ of size $d$ with no collisions and look up every entry in $A$ to find any matches. This could reduce the run time to $O(d+dn)= O(dn)$ per query.\\
These algorithms are both guaranteed to return the set $S=\max_{A \in \mathcal{F}}J(Q,A)$ on every query. The next algorithm can query much faster than this by relaxing this condition: if we allow preprocessing the data and relax the requirement of finding an exact match by considering the \textit{approximate similarity search problem}, we can create sketches of the data set before querying. This sketching strategy can drastically reduce the query time.
