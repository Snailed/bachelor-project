\subsection{Parallel Bit Counting}
To be able to filter bad matches from good matches, we can use the 1-bit minwise hashing trick to pack the results of multiple subexperiments into one word, which by comparing the bit-wise cardinality (amount of bits set in a bit vector) of two words approximates the similarity between the two. % TODO: Research this!
One of the main techniques behind achieving an efficient run time of this method requires computing the cardinality of a bit vector efficiently. \citet{fast-similarity-search} presents a parallel algorithm to perform this but does neither describe any implementation details and has an unnecessarily complicated proof. I will try to describe the same algorithm in a much simpler fashion and prove its correctness and run time while doing it.
A naive algorithm to do this could be described like in algorithm \ref{alg:naive-cardinality}.
\begin{algorithm}[H]
\caption{A naive linear time algorithm}\label{alg:naive-cardinality}
\begin{algorithmic}
\Function{Cardinality}{$w$, $d$} \Comment{$w$ is the input word, $d$ is the word-size}
\State $x \gets 0$
\State $i \gets 0$
\While{$i \leq d$}
\State $x \gets x + (w \gg i) \land 1$
\State $i \gets i + 1$
\EndWhile
\State \Return $x$
\EndFunction
\end{algorithmic}
\end{algorithm}
This algorithm trivially runs in linear time to the dimensionality $d$ of the input word $w$. For $n$ words of size $d$, this algorithm runs in $O(nd)$ time.
\citet{fast-similarity-search} presents two improvements to this: The first will improve the run time to $O(n\log(d))$ time by utilizing a divide-and-conquer technique and the second to $O(n + \log(d))$ time by introducing parallelism.
\subsubsection{Divide-and-Conquer}
To explain how divide-and-conquer methods can be used to improve run-time, we must first introduce a bit mask from \cite{fast-similarity-search} defined like so:
$$m_{i,j} = \underbrace{0\dots 0}_{2^{j}-2^{i}}\underbrace{1\dots 1}_{2^i}\dots\underbrace{0\dots 0}_{2^{j}-2^{i}}\underbrace{1\dots 1}_{2^i}$$
where $j > i$ and $j, i \in \mathbb{Z}^+$. The notation $m_{i}$ is a shorthand for $m_{i, i+1}$ and indicates a mask with an equal amount of 1's and 0's. By computing $w \land m_{i,j}$ for some word $w$ and some integers $i, j$, we can isolate a specific segments of size $2^i$ in the bitstring.
We will use this in the following operation:
\begin{equation}
    T(w, m, k) = (w\gg k) \land m
\end{equation}
This operation isolates the segments indicated by the bit-mask starting from the $k$th position of the word $w$.
The algorithm to calculate the cardinality can then be described like in algorithm \ref{alg:divide-and-conquer-cardinality}
\begin{algorithm}[H]
\caption{A Divide-and-Conquer approach}\label{alg:divide-and-conquer-cardinality}
\begin{algorithmic}
\Function{Cardinality}{$w$, $d$} \Comment{$w$ is the input word, $d$ is the word-size}
\State $i \gets 0$
\While{$i \leq \log_2(d)$}
\State $w \gets T(w, m_i, 0) + T(w, m_i, 2^i)$
\State $i \gets i + 1$
\EndWhile
\State \Return $w$
\EndFunction
\end{algorithmic}
\end{algorithm}
The simplest way to gain intuition of this algorithm is to prove it. A loop-invariant is presented like so:
\begin{invariant}
\label{thm:divide-invariant}
At the $i$th iteration of the algorithm, each bitstring-segment of size $2^i$ of the word $w_i$ will contain the cardinality of the corresponding segment of the word $w_0$. Furthermore, the operation can be done in constant time.
\end{invariant}
\begin{proof}
    When $i=0$, then each bit-string of size $2^0=1$ in $w_0 = w$ will be $1$ if the bit is $1$ and $0$ if the bit is $0$. This proves our base case. \\
    If at step $i$ the word $w_i$ contains $\frac{d}{2^i}$ segments of size $2^i$, then invariant \ref{thm:divide-invariant} will be upheld if we combine each segment with it's neighbor to form a segment of size $2^{i+1}$. Since the cardinality of a combined bitstring is equal to the cardinality of its parts, we can divide all of the segments into pairs of size $2^{i+1}$ and add them together to form the new pair, which is done by the operation $T(w, m_i, 0) + T(w, m_i, 2^i)$ in constant time if the masks are pre-computed. This operation works because $T(w, m_i, 0)$ isolates every other segment of size $2^i$ starting from the first segment and $T(w, m_i, 2^i)$ isolates every other segment of size $2^i$ starting from the second segment.\\
\end{proof}
When the algorithm terminates after $\log_2(d)$ iterations, the segments will have size $2^{\log_2{(d)}} = d$ and thus span the entire original word, which means that we have the bit-count of the original word. % TODO: Skriv noget omkring hvad der sker når d ikke er en faktor af 2
\subsubsection{Parallelism}
To introduce parallelism into the algorithm, we must first realize the following: When two segments of $2^i$ bits gets combined, they will not need all of the $2^{i+1}$ bits to represent their sum. It is actually such that the bits used at iteration $i$ is exactly $i+1$.
\begin{invariant}
    At the $i$th iteration of the algorithm, the amount of bits set in a given segment of a word is at most $i+1$.
\end{invariant}
\begin{proof}
    We will prove this by induction. \\
    At $i=0$, the size of the segments are $2^0 = 1$, which is equal to $i+1$.\\
    If it is true at iteration $i$, then at iteration $i+1$ the algorithm will add two words of size $i+1$ which creates a word of size $i+2$, which fulfills the loop invariant.
\end{proof}
Now, we will introduce a function $l(i)$, which produces the smallest number such that $2^{l(i)} \geq i + 2$. If we use the bit mask $m_{l(i), i+1}$ instead of $m_{i}$, we will get the same result. This also means that we can pack $2^{i-l(i)}$ words into one by utilizing the empty space in each segment.

