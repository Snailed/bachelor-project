% Hvad er Jaccard Similarity?
% Hvordan er problemet defineret fra et teoretisk perspektiv?
% Er der forskel i problemdefinitionerne? Hvad betyder dette for et sammenligningsperspektiv?
\subsection{Problem Definition}
The formal definition of the \textit{Jaccard Similarity Search Problem} is as follows:
Given a family $\mathcal{F}$ of $n$ sets from some universe $U$ and a query set $Q$, preprocess $\mathcal{F}$ to efficiently find the set $S\in \mathcal{F}$ such that $S = \max_{A\in \mathcal{F}}J(A,Q)$. \\
In many practical applications, it may be enough to only consider the \textit{Approximate Jaccard Similarity Search Problem} (as defined by \citet{fast-similarity-search}): Given a family $\mathcal{F}$ from a universe $U$, a query set $Q$ and two variables $0 < j_2 < j_1 < 1$, preprocess $\mathcal{F}$ such that one can efficiently find some set $A \in \mathcal{F}$ such that $J(A,Q) \geq j_2$ if there exists some set $B \in \mathcal{F}$ where $J(B,Q) \geq j_1$. Note that it is possible for $A = B$. The intuition to this is that we can dial in how precise we expect our algorithm to be able to solve this problem by fine-tuning $j_2$ and $j_1$. We no longer expect the algorithm to return the \textit{best} solution, just one that is "good enough" (if it exists). Algorithms that solve this problem should also be able to return "no answer", if no set $B\in \mathcal{F}$ upholds $J(B,Q)\geq j_1$, which is entirely possible depending on the parameters. \\
Most existing methods to solve this problem depend on the relationship between $j_1$ and $j_2$. This relationship is often described as $\rho=\frac{\log_2(1/j_1)}{\log_2(1/j_2)}$. Note that $\rho$ shrinks when $j_1$ approaches $1$.

