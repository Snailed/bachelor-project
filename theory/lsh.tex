\subsection{Locality Sensitive Hashing Framework}
One of the first big contributions to this problem is the Locality Sensitive Hashing Framework by \citet{Indyk1998ApproximateNN}. This framework aims to solve the Approximate Similarity Search problem for any kind of similarity space, including Jaccard similarity, using some appropriate $(s_1, s_2, r_1, r_2)$-sensitive family of hash functions.
\begin{definition}
\label{thm:sensitive-hash}
A family of hash functions $\mathcal{H}=\{h\colon X\rightarrow U\}$ is $(s_1, s_2, r_1, r_2)$-sensitive for some set $X$ and some universe $U$ with a similarity function $S\colon X\times X \rightarrow [0;1]$ if these conditions apply for any $x,y \in X$ and any $h \in \mathcal{H}$:
\begin{itemize}
    \item If $S(x,y) \geq s_1$, then $Pr[h(x)=h(y)] \geq r_1$
    \item If $S(x,y) \leq s_2$, then $Pr[h(x)=h(y)] \leq r_2$
\end{itemize}
\end{definition}
First, we define a family of functions $\mathcal{G}=\{g\colon X \rightarrow U^k \}$ such that for each $g\in \mathcal{G}$, $g(p)=\langle h_0(p), \dots, h_{k-1}(p)\rangle$ for $h_i\in \mathcal{H}$, where $k\in \mathbb{Z}^+$ is to be determined later. We can now regard $g(p)$ as the index of a ``bucket'', where we can store the value $p$. We do this for each of the functions $g_0, \dots, g_{l-1} \in \mathcal{G}$ for each data point $p\in X$ and some value $l\in \mathbb{Z}^+$ to be determined later. If a collision happens, we store only one of the colliding points, chosen at random.\\
To query, look up the corresponding point for each of the buckets $g_0(q), \dots, g_{l-1}(q)$. This yields the points $P=\{p_0, \dots, p_{t-1}\}$ where $t\leq l$ is the amount of points found. For each $p_i \in P$, if $S(p_i, q) \geq j_2$, then return it. If no points fulfills this constraint, return false.\\
Part of implementing this framework for a specific similarity metric (e.g. Jaccard Similarity) requires choosing a suitable value for $k$ and $l$. Finding the right parameters can in this case result in getting a constant error probability, as we will show later.
