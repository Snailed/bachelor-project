\subsection{Correctness}
% Hvordan ved jeg at min kode er korrekt? DONE
% Kan jeg udregne alle eksempler i hånden? DONE
% Hvor meget af koden har jeg testet? Randomiseret input vs håndkørt input. DONE
It is luckily easy to verify the correctness of the implementation of the algorithm since it is a deterministic algorithm. That means that it is possible to calculate any expected output by hand, which both is useful for validating and for debugging.\\
Which inputs should be used though? To be able to show that the implementation works on a wide array of inputs, we can generate a large list of random numbers to help cover eventual edge cases. We can then compare the results to some reference implementation - in this case the built-in \texttt{count\_ones()} method.\\
% Hvordan kan jeg vide at koden virker med en stor ordstørrelse?
It is much harder to verify whether the code works with multiple different word-sizes. This is due to the fact that the else branch on line 16 of algorithm \ref{alg:parallel-d-and-c} only gets hit once when working with $d=64$ (when $i=2$) and twice when $d=128$. Since the applications of this algorithm really benefits when using large word sizes, it is very relevant that this gets tested. A word size of $d=256$ was used as well to show that the else-statement path as described earlier also works as intended.
