% Hvorfor foretage teoretisk køretidsanalyse? Hvorfor lave en praktisk implementation?
% Har vi en hypotese? Hvordan kan vi statisk be/afkræfte denne?
% Hvad kan vi realistisk forvente at en implementation kan fortælle os?
% Hvorfor benchmarke? Er benchmarks reproducérbare? Hvordan benchmarker man en monte carlo algoritme?
% Hvis vi forkaster vores nulhypotese, hvad kan vi så fortælle omkring vores resultater?

\subsection{Methods}
% % 1. Fast Similarity Searching er markant hurtigere til søgning end andre eksisterende løsninger, selv med parametre der fører til en acceptabel præcision.
% 2. Fast Similarity Searching er præcis nok til at blive praktisk anvendt.
\subsection{Hypothesis}

For the rest of the project, the main focus will be on showing the practical feasability of the parallel bit-counting algorithm compared to simpler algorithms. 
Since $d$ needs to be very large for parallelism to affect the results, it can be very hard to compare the algorithms presented in the previous section in a meaningful way if multiple bit-strings should be embedded in any word-sizes found on a standard machine. For the rest of the project, it therefore be assumed that $d=w$, which means that only 1 bit-string will be embedded per word.\\
We know from the time complexity of the algorithms in the previous section that the parallel algorithm gets theoretically better than the other algorithms as $d$ grows. If we find out that the parallel algorithm is slower than the other algorithms in the case that $d=w$, then we can assume that it is also slower when $d \leq w$. If we however find out that the parallel algorithm is faster, then we might be unable to make any conclusions on the algorithms feasibility, since its usefulness in the filtering step of the querying algorithm relies on its ability to work on words with multiple embedded bit-strings. \\
A comparison will be made to the \texttt{popcnt} CPU instruction that is implemented on most x64-86 CPUs, which can count the bit-wise cardinality of a word\cite{popcnt}.
% Hvad fortæller benchmarking os om noget som helst?
% Hvordan kan vi statistisk vide noget om benchmarking?
% Vil resultaterne være reproducérbare? Hvordan kan vi vide dette? Bør vi køre koden på flere maskiner?
\subsubsection{Benchmarking}
To be able to quantify whether one algorithm is faster than the other, one can try to implement it in practice, run it on a specified data set and measure how long it takes for the algorithm to terminate. Benchmarking this way is common in the industry, but it can be hard to be able to generalize from results, due to the many possible sources of error such as:
\begin{itemize}
    \item Results can vary from machine to machine due to factors like difference in instruction sets, processor infrastructure, cache hierarchy, compiler compatibility, memory capacity etc.
    \item The input data set can be biased towards one specific kind of architecture or implementation.
    \item Caches become more efficient when ``warmed up'' e.g. when the computer has recently fetched from the same memory addresses. Two identical benchmarks might produce different results based on whether or not the caches are filled.
    \item A process that benchmarks a program will have to share the CPU cores with other processes running on the computer. This can produce a high variance between identical benchmarks on the same machine.
    \item Different compilers can produce more or less optimized code, which might produce varying results on different compilation targets.
\end{itemize}
The scope of this project disallows running benchmarks on a wide variety of computers, which presents some uncertainty in the validity of the results. Some of the other sources of error are mitigable.
\begin{itemize}
    \item The benchmark should be run multiple times with different pseudo-randomly generated input to mitigate bias in the input data.
    \item The benchmark should be run multiple times per input data set, until a significant mean execution time can be calculated.
    \item The benchmark should perform a few "dry runs" before each run to warm up the caches.
\end{itemize}
This applications of this software lends itself to large-scale machine learning applications, search engines and data mining. Therefore, the most relevant results would come from running the benchmarks on a machine with access to a GPU, and for the algorithm to be written with parallelism\footnote{In this case, parallelism is meant in a GPU context unlike word parallelism as mentioned previously} in mind. This has not been considered within the scope of this project however, but instead lends itself as an open research problem.

