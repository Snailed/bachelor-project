% Vi foretog indgående analyse af eksisterende løsninger på Approximate SS problemet, der viste at advancerede teknologier som præsenteret af Dahlgaard et al. osv er teoretisk overlegende hvis man er villig til at foretage nogle specifikke tradeoffs.
% En praktisk implementation viste sådan og sådan med denne specificitet ud fra sådanne benchmarks.
% Vores konklusioner er bla bla, og det indikerer bla bla.
% Ydergående undersøgelser kan undersøge ligende og lignende.
\section{Conclusion}
An in-depth analysis of existing solutions to the Approximate Similarity Search Problem described how many existing solutions utilize sketching to achieve a low query time within a constant error probability. New research by \citep{fast-similarity-search} shows how to achieve an even better error probability by segmenting a large sketch into multiple sections and regarding them as semi-independant experiments. An important part of this optimization relies in being able to quantify the results of each experiment very efficiently - in this case by counting the cardinality of a bit-string, which \citep{fast-similarity-search} presents a presumably very efficient algorithm to do.\\
An analysis of the algorithm shows its correctness with a few modifications alongside a theoretical running time which matches \cite{fast-similarity-search}. Benchmarks performed on an actual application tells otherwise however, showing that the word parallelism used to achieve a low amortized cost actually has a too high complexity to match simpler algorithms in practice when used on standard word-sizes. Further research might focus on benchmarking the algorithm on more specialized hardware, although rewritting the implementation seems to be non-trivial.
