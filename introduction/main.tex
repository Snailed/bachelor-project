\section{Introduction}
The \textit{Approximate Similarity Search Problem} regards efficiently finding a set $A\in \mathcal{F}$ from some corpus $\mathcal{F}$ that is approximately similar to a query set $Q$ in regards to the \textit{Jaccard Similarity} metric $J(A,Q) = \frac{|A\cap Q|}{|A\cup Q|}$\cite{dahlgaard2017fast}\cite{fast-similarity-search}. Approximate means that given two variables $0 < j_2 < j_1 < 1$, the algorithm should efficiently find some set $A\in \mathcal{F}$ such that $J(A, Q) \geq j_2$ if there exists some set $B\in \mathcal{F}$ where $J(B,Q)\geq j_1$.
Practical applications includes searching through large corpi of high-dimensional text documents like plagiarism-detection or website duplication checking among others\cite{vassilvitskii2018}. The main bottleneck in this problem is the \textit{curse of dimensionality}. 
A trivial algorithm can solve this problem in $O(n|Q|\max_{A\in \mathcal{F}}{|A|})$ time (where $n$ is the amount of elements in the corpus $\mathcal{F}$), but algorithms with a query-time proportional to the dimensionality of the corpus $\max_{A\in \mathcal{F}}|A|$ scale poorly when working with high-dimensional datasets. 
Text documents are especially bad in this regard since they often are encoded using \textit{$w$-shingles} ($w$ contigous words) which \citet{li2011hashing} shows easily can reach a dimensionality upwards of $|Q|=2^{83}$ using just $5$-shingles.\\
The classic solution to this problem is the MinHash algorithm presented by \citet{broder1997minhash} to perform website duplication checking for the AltaVista search engine. 
It preprocesses the data once using hashing to perform effective querying in $O(n^\rho\log(n)|Q|)$ time (where $\rho = \frac{\log_2{(1/j_1)}}{\log_2{(1/j_2)}}$), a significant improvement independent of the dimensionality of the corpus.
Many improvements have since been presented to both improve preprocessing time, query time and space efficiency. 
Notable mentions includes (but are not limited to) the use of \textit{tensoring}\cite{andoni2006efficient}, \textit{b-bit minwise hashing}\cite{ping2011theory}, \textit{fast similarity sketching}\cite{dahlgaard2017fast}. 
Applications of these techniques lead to efficient querying with a constant error probability. If one wishes to achieve an even better error probability such as $\varepsilon = o(1)$, it is standard practice within the field to use $O(\log_2(1/\varepsilon))$ independent data structures and return the best result, resulting in a query time of $O(\log_2(1/\varepsilon) (n^\rho + |Q|))$. 

Recent advances by \citet{fast-similarity-search} show that it is possible to achieve an even better query time by sampling these data structures from one large sketch. The similarity between these sub-sketches and a query set needs to be evaluated efficiently when querying, which requires efficient computation of the cardinality of a bit-string. To do this, \citet{fast-similarity-search} presents a general parallel bit-counting algorithm that computes the cardinality of a list of bit-strings in sublinear time amortized due to word-parallism. This, combined with an efficient filtering scheme, brings the query time down to $O((\frac{n\log_2 w}{w})^\rho \log(1/\varepsilon) + |Q|)$.\\
The main focus of this project is to analyse, prove, implement and evaluate this parallel bit-counting technique. The analysis will be based on the original paper \cite{fast-similarity-search}, but with some modifications to resolve some of the issues with the original algorithm. This will also include a pseudo-code implementation of the algorithm since the original paper only describes it through recurrences. This leads to a proof of correctness that slightly differs from the one presented in the paper, and a time complexity analysis that does indeed show the sublinear running time as claimed.\\
The theoretical analysis will be backed up by a real-life implementation that can be benchmarked to help show this sublinear running time in practice. At last, reflections on the results and methods will be made to back up eventual conclusions.

