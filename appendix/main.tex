\begin{appendices}
\section{Machine Specifications}
\label{appendix:machine-specs}
\begin{table}[H]
\begin{tabular}{ll}
\hline
\multicolumn{1}{|l|}{Model}            & \multicolumn{1}{l|}{Lenovo Thinkpad E580}                        \\ \hline
\multicolumn{1}{|l|}{Operating System} & \multicolumn{1}{l|}{Arch Linux kernel 5.15.5-arch1-1}            \\ \hline
\multicolumn{1}{|l|}{Word size}        & \multicolumn{1}{l|}{64bit}                                       \\ \hline
\multicolumn{1}{|l|}{Processor model}  & \multicolumn{1}{l|}{Intel(R) Core(TM) i7-8550U CPU @ 1.80GHz}    \\ \hline
\multicolumn{1}{|l|}{ISA}              & \multicolumn{1}{l|}{x86\_64}                                     \\ \hline
\multicolumn{1}{|l|}{Memory Capacity}              & \multicolumn{1}{l|}{32 GB}                                     \\ \hline
\multicolumn{1}{|l|}{Caches}           & \multicolumn{1}{l|}{128 KiB L1d, 128 KiB L1i, 1MiB L2, 8 MiB L3} \\ \hline
                                       &                                                                  \\
                                       &                                                                  \\
                                       &                                                                  \\
                                       &                                                                  \\
                                       &                                                                  \\
                                       &                                                                  \\
                                       &                                                                 
\end{tabular}
\end{table}

\section{Code}
\label{appendix:code}
The code for the implementation part can be found on in this Github repository: \url{https://github.com/Snailed/parallel-bit-counting}.
\end{appendices}
