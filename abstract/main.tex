% Spørgsmål:
% Hvilket problem omhandler denne artikel?
% Hvad prøver jeg at undersøge ved dette problem?
% Hvordan undersøger jeg dette?
% Hvad var mine konklusioner?
% Hvor sikre var disse?
\begin{abstract}
    % This project regards an analysis, implementation and comparison of existing efficient solutions to the Approximate Jaccard Similarity Search Problem of estimating the Jaccard Similarity between multiple sets. The findings include both a large theoretical and practical advantage of using advanced methods as presented by Dahlgaard et al.\cite{dahlgaard2017fast} and Knudsen\cite{fast-similarity-search} if the user is willing to pre-process the search-corpus before performing the search. This comparison both regards the precision of different methods as well as the running time of an real-world implementation. This is based on an implementation written in a low-level systems language and benchmarked on a regular personal computer with a statistical significance. Furthermore, reflections are made on how the results might scale on more specialized hardware.
    This project regards an analysis of recent advances in solving the \textit{Approximate Similarity Search Problem} with respect to the \textit{Jaccard Similarity} $J(A,B)=\frac{A\cap B}{A\cup B}$, for some subsets $A,B$ of some universe $U$. Given a corpus $\mathcal{F}$ consisting of $n$ subsets of $U$, two constants $0 < j_2 < j_1 < 1$ as well as a set $Q\subseteq U$ find a set $A\in \mathcal{F}$ such that $J(A,Q) \geq j_2$ if there exists a set $B\in \mathcal{F}$ where $J(B,Q) \geq j_1$.\\
    First, existing solutions to this problem will be described, such as the \textit{Locality Sensitive Hashing Framework} (LSH) by \citet{classic-lsh}, the seminal \textit{MinHash} algorithm by \citet{broder1997minhash} and the \textit{Fast Similarity Sketching} algorithm by \citet{dahlgaard2017fast}.
    Then the focus will be shifted to the most recent advances by \citet{fast-similarity-search}, who presents a parallel bit-counting algorithm that supposedly can make the query time sub-linear to the amount of sets in $\mathcal{F}$. \\
    The efficiency of this bit-counting algorithm will then be investigated by evaluating the algorithm in terms of time complexity, proof of correctness and an empirical comparison to simpler methods.
    The findings include a theoretical time advantage to using parallel bit counting compared to a simple, linear time algorithm, but empirical results show that this advantage is hard to achieve in practice. 
    Furthermore, reflections are made on how these results might scale on more specialized hardware, wherein the Approximate Similarity Search Problem may find the most use.
\end{abstract}
